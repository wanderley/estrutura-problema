\section*{Problema \proxLetra: NOME DO PROBLEMA}
\arquivoProblema{nome}

% Para adicionar uma imagem ou incluir um arquivo .tex você precisa 
% adicionar \PASTA como prefixo do diretório. Essa variável conterá o 
% caminho relativo (ao documento principal) do diretório.
% 
% Exemplo:
% \begin{figure}
%   \includegraphics{\PASTA/imagens/exemplo.pdf}
% \end{figure}

HISTÓRIA

\subsection*{Entrada}
% Retire o comentário da linha abaixo para adicionar um texto dizendo 
% que sua entrada é composta por diversas instâncias.
% \textoDiversasInstancias

A linha 1 contém bla bla

A linha 2 contém bla bla

 
\subsection*{Saída}

Para cada instância imprima ..

\subsection*{Restrições}
\begin{itemize}
  \item $1 \leq N \leq 100$

  \item $1 \leq N \leq 100$
\end{itemize}

\subsection*{Exemplos}

\exemplo{tests/in1}{../tests/out1}
\exemplo{tests/in2}{../tests/out2}

% vim: set nocindent formatoptions+=aw tw=72:
